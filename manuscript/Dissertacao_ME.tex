%%%Preamble
% packages and document class
\documentclass[a4paper, 12pt]{article}
\usepackage[utf8]{inputenc}
\usepackage{geometry}
\usepackage{authblk}

\bibliography{master_mutantag.bib}

% page format
\geometry{left = 3cm, top = 3cm, bottom = 2cm, right = 2cm}
\linespread{1.5}

% article basic infos
 \title{\line(1,0){250}\\From mutualism to antagonism: the coevolutionary influence of  context-dependent interactions in mutualistic networks\\\line(1,0){250}}

\author[1]{Lucas A. Camacho}
\author[2]{Paulo Roberto Guimarães Junior}

\affil[1,2]{Departamento de Ecologia, Universidade de São Paulo, Rua do Matão, travessa 14, nº 321, Cidade Universitária, São Paulo - SP, CEP: 05508-090, Brasil.}

%%%Document begins
\begin{document}
\maketitle

\section{Introduction}
Coevolution, the reciprocal evolutionary change between interacting species, is a main force influencing the diversity of species traits and the organization of ecological interactions in the community. 
The interactions structure dictates which species are having or not the reciprocal evolutionary change, or coevolving in the community. 
In this way, coevolution is a process that molds and is molded by ecological interactions in the community. 
The most conspicuous patterns known of coevolution are on species traits related to ecological interactions like plants and herbivores, pollination or seed dispersal.

\par The empirical evidences of coevolution thrilled several worldwide known naturalists. 
Charles Darwin pointed out  the orchid \textit{Angraecum sesquipedale} and their 30 centimeters floral nectaries and predicts that \textit{A. sesquipedale} pollinator should have a proboscis with a similar size. 
Years latter, the moth that pollinates the orchid, \textit{Xantophan morgani}, was described and Darwin's prediction was correct. 
Also, Fritz Müller studied the coloration patterns of neotropical butterflies and how these patterns emerge by coevolution, proposing the first mathematical model for coevolution. 
The empirical evidences of coevolution lead to the description of genetic and ecological mechanisms influencing the coevolution and the ecological interactions between species in communities.

\par Coevolution process depends on how the ecological interactions are distributed in the community.
An possible approach to explore the coevolution is using networks theory. Networks are representations of species and the interactions between these species in the community.
The use of networks of interactions enable the investigation of how different evolutive process form phenotipic pattern of species. forming a new research line called ecological networks
Using the networks approach, we now know that coevolution in mutualistic networks of interactions lead to trait complementarity of species that interact. In antagonisms otherwise, the selection intensity acting on a prey and the predator can create coevolutionary arm's race. 
This different coevolutionary dynamics can reorganize the interactions structure in time, generating for example, temporal variation in species trait assimetry between interacting species.

\par The assimetry level, the interaction network structure and the temporal variation of coevolution rely on the costs and benefits associated with different interaction outcomes.
For example, mutualisms shows a higher benefit compared to the cost for both species compared with antagonism that has a higher benefit compared with the cost for an individual of a predator or parasite specie and a low benefit than the cost for the prey or host.
Despite the utility of using the interactions by their costs and benefits, these costs and benefits are not fixed.
The variation of benefits and costs happens on the biotic and abiotic which the species are under.
The interactions outcomes which vary because of biotic and abiotic factors are called context-dependent interactions.

\par There is growing evidence quantifying the outcomes variation of interactions in space and time.
For example, mirmecophyte plants has structures called domatia and extrafloral nectaries which atract ants. 
These ants repeal natural enemies of plants like herbivores. 
But, a low abundance of ants caused by external factors of the plant can cause a low repealing efficiency from ants. 
In this scenario, is possible that the production cust of domatia and extrafloral nectaries for the plant could be lower than the benefits gived by the ants, for the low efficiency in repealing natural enemys of these ants. 
In this way, the interaction between mirmecophyte plants and ants can pass from a mutualism to an antagonism, which the ant is beneficied and the plant suffers from a higher cust, depending on the ecological context.

\par Depending on the interaction, this interaction outcome shift from mutualism to antagonism can happen in time. 
In this way, the "come and go" of mutualists and antagonism oucomes in a community result in coevolutionary dynamics favored by these two types of interactions. 
Both dynamics in the same community can influence species more in a mutualism or antagonism-like dynamic. 
In other words, the context dependency of interactions changing the interactions outcomes generates changes in the coevolutionary dynamics of species. 
In a general view, ecological interactions outcomes varying in space and time can change the coevolutive process, contributing to the trait diversity of species and interaction structure of the community. 
Finally, context-dependent interactions should not be ignored if we want a higher understanding of the ecossistems function and diversity. 

\section{References}

\end{document}