%%%Preamble
% packages and document class
\documentclass[a4paper, 12pt]{article}
\usepackage[utf8]{inputenc}
\usepackage{geometry}
\usepackage{authblk}

% page format
\geometry{left = 3cm, top = 3cm, bottom = 2cm, right = 2cm}

% article basic infos
\title{Fom mutualism to antagonism: the coevolutionary influence of  context-dependent interactions in mutualistic networks }

\author[1]{Lucas A. Camacho}
\author[2]{Paulo Roberto Guimarães Junior}

\affil[1,2]{Departamento de Ecologia, Universidade de São Paulo, Rua do Matão, travessa 14, nº 321, Cidade Universitária, São Paulo - SP, CEP: 05508-090, Brasil.}

%%%Document begins
\begin{document}
\maketitle

\section{Introduction}
Coevolution, the reciprocal evolutionary change between interacting species, is a main force influencing the diversity of species traits and the organization of ecological interactions in the community \cite{chamberlain_how_2014}. The interactions structure dictates which species are having or not the reciprocal evolutionary change, or coevolving in the community. In this way, coevolution is a process that molds and is molded by ecological interactions in the community. The most conspicuous patterns known of coevolution are on species traits related to ecological interactions like plants and herbivores, pollination or seed dispersal.

\par The empirical evidences of coevolution thrilled several worldwide known naturalists. Charles Darwin pointed out  the orchid \textit{Angraecum sesquipedale} and their 30 centimeters floral nectaries and predicts that \textit{A. sesquipedale} pollinator should have a proboscis with a similar size. Years latter, the moth that pollinates the orchid, \textit{Xantophan morgani}, was described and Darwin's prediction was correct. Also, Fritz Müller studied the coloration patterns of neotropical butterflies and how these patterns emerge by coevolution, proposing the first mathematical model for coevolution. The empirical evidences of coevolution lead to the description of genetic and ecological mechanisms influencing the coevolution and the ecological interactions between species in communities.

\par Ecological interactions are one of the main forces that influences the survivorship and reproduction of species individuals in ecological communities. Different interaction signals (positive or negative) have different outcomes in life history of species individuals. For example. antagonist interactions like predation or parasitism have a positive effect on survival and/or reproduction for the population of predator or parasite and a negative effect for the  population of prey or host. Differently, mutualism interactions have a positive effect on populations of species that interact. Depending on the outcomes of these interactions and which species are interacting, species can have different evolutionary pathways of trait changing (REF). In other words, we can say that the different interaction outcomes influences differently the coevolution on the community. 

\par Also, interactions outcomes are not fixed but can change in space and time depending on factors like abundance of species individuals (REF), climate change (REF), resource quantity or quality (REF), etc. The variation on interaction outcomes in time change how species reciprocally influences each other (Figure 1).  


\section{References}

\end{document}